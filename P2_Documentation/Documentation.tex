\documentclass[]{article}
%----------------- PAKETE INKLUDIEREN ----------------- %

\usepackage{geometry} % Packet für Seitenrandabständex und Einstellung für Seitenränder
\usepackage[ngerman]{babel} % deutsche Silbentrennung

\usepackage{booktabs} %entzerrt die Tabellenzeilen und bietet verschieden dicke Unterteilungslinien
\usepackage{longtable} % Tabellen können sich nicht über mehrere Seiten 
\usepackage{graphicx} % kann LaTeX Grafiken einbinden

%\usepackage[applemac]{inputenc} % Umlaute unter Mac werden automatisch gesetzt
\usepackage[T1]{fontenc} % Zeichenencoding
\usepackage[utf8]{inputenc}
\usepackage{lmodern} % typographische Qualität 
\frenchspacing % Schaltet den zusätzlichen Zwischenraum ab
\usepackage{fix-cm}
\usepackage{hyperref} % verwandelt alle Kapitelüberschriften, Verweise aufs Literaturverzeichnis und andere Querverweise in PDF-Hyperlinks
\usepackage{color}
\usepackage{url}

\usepackage[nottoc]{tocbibind}
\usepackage{subfigure}
\usepackage{float}
\usepackage{mathtools}
% für Listings
\usepackage{listings}
\lstset{numbers=left, numberstyle=\tiny, numbersep=5pt, stepnumber=4, keywordstyle=\color{black}\bfseries\itshape, stringstyle=\ttfamily,showstringspaces=false,basicstyle=\footnotesize,captionpos=b}
\lstset{language=java}
%Todos
\usepackage{todonotes} % to erase all comments by adding [disable]

%opening
\title{Ausarbeitung zu Chaos und Faktale Praktikum 1}
\author{Jannis Priesnitz\space \textperiodcentered \space Margarethe Dziendziel}

\begin{document}

\maketitle
\begin{center}

\end{center}
\section{Ausdruck Ihrer IFS-Datei IFS\_TEST.IFS aus der zweiten Teilaufgabe. }

\begin{tabular}{cccccc}
	3 &  &  &  &  &  \\ 
	0.5 & 0.0 & 0.0 & 0.5 & 0.0 & 0.0 \\ 
	-0.5 & 0.0 & 0.0 & 0.5 & 1.0 & 0.0 \\ 
	0.0 & 0.5 & -0.5 & 0.0 & 0.5 & 1.0 \\ 
	-0.1 & 1.1 & -0.1 & 1.1 &  &  \\ 
	1 & 1 &  &  &  & 
\end{tabular} 
\todo{ist das das richtige?}

\section{Was ergab sich in der ersten Teilaufgabe (a) beim Test  unterschiedlicher Anfangsmengen und nach welchem „Satz“ war dies nicht anders zu erwarten? }
Es ergeben sich immer wieder Sierpinski-Dreiecke. Dies ergbit sich aus dem verwendeten Iteriertem Funktionen System "`Sierpinski.ifs"'.
\section{Was ergab sich beim Test von x2.bmp und was stellt der Inhalt dieses Bildes bzgl. des IFS folglich dar? (Tipp: siehe Skriptum Kapitel 3 Seite 2). }
Es ergab sich das bereits bei den anderen Lösungen gesehen Sierpinski-Dreieck, diesmal jedoch mit immer kleiner werdenden Bildern. An diesen Bildern können die affinen Transformationen, Skallierung, Rotation, Translation erkannt werden. \todo[inline]{Oder?}
\section{Zeigen Sie rechnerisch nachvollziehbar, dass das Chaos-Spiel aus der ersten Aufgabe mit den  drei  homogenen  Eckpunkten   [0,0,1]t , [1,0,1]t und  [0,1,1]t für  einen  allgemeinen  Punkt P=[xP,yP,1]t genau dasselbe bewirkt, wie die drei Transformationen in SIERPINSKI.IFS.}

Wir zeigen rechnerisch an einem Beispiel, dass die Ergebnisse für $P=[0.5,0.5,1]^t$ identisch sind. 

\subsubsection*{IFS-Methode}
Eine "`Rückwärtstranformation"' von "`Sierpinski.ifs"' ergibt folgende drei Matrizen:
 \[  
 A = 
 \begin{pmatrix}
 0.5 & 0 & 0 \\
 0 & 0.5 & 0 \\
 0 & 0 & 1 \\
 \end{pmatrix}
 ,
 B = 
 \begin{pmatrix}
 0.5 & 0 & 0.5 \\
 0 & 0.5 & 0 \\
 0 & 0 & 1 \\
 \end{pmatrix}
,
 C = 
 \begin{pmatrix}
 0.5 & 0 & 0 \\
 0 & 0.5 & 0.5 \\
 0 & 0 & 1 \\
 \end{pmatrix}
 \] 
 
Multiplikation der Transformationsmatrizen mit dem gewählten Punkt ergibt: 
 \[  
 A = 
 \begin{pmatrix}
 0.5 & 0 & 0 \\
 0 & 0.5 & 0 \\
 0 & 0 & 1 \\
 \end{pmatrix}
 * 
 \begin{pmatrix}
 0.5\\
 0.5\\
 1 \\
 \end{pmatrix} 
 =
 \begin{pmatrix}
 0.25\\
 0.25\\
 1 \\
 \end{pmatrix} 
    \] \[
 B = 
 \begin{pmatrix}
 0.5 & 0 & 0.5 \\
 0 & 0.5 & 0 \\
 0 & 0 & 1 \\
 \end{pmatrix}
  * 
  \begin{pmatrix}
  0.5\\
  0.5\\
  1 \\
  \end{pmatrix} 
  =
  \begin{pmatrix}
  0.75\\
  0.25\\
  1 \\
  \end{pmatrix} 
  \] \[
 C = 
 \begin{pmatrix}
 0.5 & 0 & 0 \\
 0 & 0.5 & 0.5 \\
 0 & 0 & 1 \\
 \end{pmatrix}
  * 
  \begin{pmatrix}
  0.5\\
  0.5\\
  1 \\
  \end{pmatrix} 
  =
  \begin{pmatrix}
  0.25\\
  0.75\\
  1 \\
  \end{pmatrix} 
 \] 
\subsubsection*{Chaos-Spiel-Metode}
Hier betrachten wir die Fälle, dass der Zufallsgenerator 0, 1 oder 2 "`würfelt"':\\
1. Random = 0:   
\[ 
Punkt:
\begin{pmatrix}
0 \\
0 \\
1 \\
\end{pmatrix}
:
\begin{pmatrix}
(0 + 0.5) \div 2 \\
(0 + 0.5) \div 2 \\
1 \\
\end{pmatrix}
=
 \begin{pmatrix}
 	0.25\\
 	0.25\\
 	1 \\
 \end{pmatrix} 
 \]
  2. Random = 1: 
 \[ 
 Punkt:
 \begin{pmatrix}
 1 \\
 0 \\
 1 \\
 \end{pmatrix}
 : 
 \begin{pmatrix}
 (1 + 0.5) \div 2 \\
 (0 + 0.5) \div 2 \\
 1 \\
 \end{pmatrix}
 =
 \begin{pmatrix}
 0.75\\
 0.25\\
 1 \\
 \end{pmatrix} 
 \]
 3. Random = 2:   
 \[ 
  Punkt:
  \begin{pmatrix}
  0 \\
  1 \\
  1 \\
  \end{pmatrix}
  : 
 \begin{pmatrix}
 (0 + 0.5) \div 2 \\
 (1 + 0.5) \div 2 \\
 1 \\
 \end{pmatrix}
 =
 \begin{pmatrix}
 0.25\\
 0.75\\
 1 \\
 \end{pmatrix} 
 \]
 
 Da bei beiden Rechnungen das selbe Ergebnis herauskommt bewirken beide Verfahren das selbe.
 \todo[inline]{Keinen PLan ob das irgendwas mit dem zu tun hat, was er sehen will - was besseres ist mir nicht eingefallen.}
\section{Geben Sie zu DECKCHEN\_3\_4TEL.IFS je Transformation des IFS die Werte der Skalierung, Rotation und Translation an. }
\subsubsection*{1. Zeile}
\[
T_S = \begin{pmatrix}
 	0.5 & 0 & 0 \\
 	0 & -0.5 & 0 \\
 	0 & 0 & 1 \\
 \end{pmatrix},
 T_R = \begin{pmatrix}
 cos(0) & -sin(0) & 0 \\
 sin(0) & cos(0) & 0 \\
 0 & 0 & 1 \\
 \end{pmatrix},
 T_T = \begin{pmatrix}
 1 & 0 & 0 \\
 0 & 1 & 0.5 \\
 0 & 0 & 1 \\
 \end{pmatrix}
 \]
 
 \subsubsection*{2. Zeile}
\[
T_S = \begin{pmatrix}
0.5 & 0 & 0 \\
0 & 0.5 & 0 \\
0 & 0 & 1 \\
\end{pmatrix},
T_R = \begin{pmatrix}
cos(0) & -sin(0) & 0 \\
sin(0) & cos(0) & 0 \\
0 & 0 & 1 \\
\end{pmatrix},
T_T = \begin{pmatrix}
1 & 0 & 0 \\
0 & 1 & 0.5 \\
0 & 0 & 1 \\
\end{pmatrix}
\]
 
 \subsubsection*{3. Zeile}
\[
T_S = \begin{pmatrix}
-0.5 & 0 & 0 \\
0 & 0.5 & 0 \\
0 & 0 & 1 \\
\end{pmatrix},
T_R = \begin{pmatrix}
cos(0) & -sin(0) & 0 \\
sin(0) & cos(0) & 0 \\
0 & 0 & 1 \\
\end{pmatrix},
T_T = \begin{pmatrix}
1 & 0 & 1 \\
0 & 1 & 0.5 \\
0 & 0 & 1 \\
\end{pmatrix}
\]
 
 \subsubsection*{4. Zeile}
\[
T_S = \begin{pmatrix}
0.333 & 0 & 0 \\
0 & 0.333 & 0 \\
0 & 0 & 1 \\
\end{pmatrix},
T_R = \begin{pmatrix}
cos(0) & -sin(0) & 0 \\
sin(0) & cos(0) & 0 \\
0 & 0 & 1 \\
\end{pmatrix},
T_T = \begin{pmatrix}
1 & 0 & 0.333 \\
0 & 1 & 0.333 \\
0 & 0 & 1 \\
\end{pmatrix}
\]
\section{ Erstellen Sie eine alternative IFS-Datei, die dasselbe Ergebnis wie DECKCHEN\_3\_4TEL.IFS erzeugt, aber mindestens zwei Rotationen (ungleich 0 und ungleich 180) enthält!}
\subsubsection*{1. Zeile}
\[
T_S = \begin{pmatrix}
0.5 & 0 & 0 \\
0 & 0.5 & 0\\
0 & 0 & 1 \\
\end{pmatrix},
T_R = \begin{pmatrix}
cos(270) & -sin(270) & 0 \\
sin(270) & cos(270) & 0 \\
0 & 0 & 1 \\
\end{pmatrix},
T_T = \begin{pmatrix}
1 & 0 & 0 \\
0 & 1 & 0.5 \\
0 & 0 & 1 \\
\end{pmatrix}
\]

\subsubsection*{2. Zeile}
\[
T_S = \begin{pmatrix}
0.5 & 0 & 0 \\
0 & 0.5 & 0 \\
0 & 0 & 1 \\
\end{pmatrix},
T_R = \begin{pmatrix}
cos(0) & -sin(0) & 0 \\
sin(0) & cos(0) & 0 \\
0 & 0 & 1 \\
\end{pmatrix},
T_T = \begin{pmatrix}
1 & 0 & 0 \\
0 & 1 & 0.5 \\
0 & 0 & 1 \\
\end{pmatrix}
\]

\subsubsection*{3. Zeile}
\[
T_S = \begin{pmatrix}
0.5 & 0 & 0 \\
0 & 0.5 & 0 \\
0 & 0 & 1 \\
\end{pmatrix},
T_R = \begin{pmatrix}
cos(90) & -sin(90) & 0 \\
sin(90) & cos(90) & 0 \\
0 & 0 & 1 \\
\end{pmatrix},
T_T = \begin{pmatrix}
1 & 0 & 1 \\
0 & 1 & 0.5 \\
0 & 0 & 1 \\
\end{pmatrix}
\]

\subsubsection*{4. Zeile}
\[
T_S = \begin{pmatrix}
0.333 & 0 & 0 \\
0 & 0.333 & 0 \\
0 & 0 & 1 \\
\end{pmatrix},
T_R = \begin{pmatrix}
cos(0) & -sin(0) & 0 \\
sin(0) & cos(0) & 0 \\
0 & 0 & 1 \\
\end{pmatrix},
T_T = \begin{pmatrix}
1 & 0 & 0.333 \\
0 & 1 & 0.333 \\
0 & 0 & 1 \\
\end{pmatrix}
\]
\todo[inline]{bitte noch mal gegenchecken}
\section{(Wie lässt sich die Gray-Figur aus den Zusätzen der 1. Praktikums-Aufgabe per IFS erzeugen? Bitte IFS-Datei in den Ausdruck einfügen.)} 
\section{Ausdruck Ihrer inversen IFS-Datei, die den in Teilaufgabe 2 gezeigten Attraktor erzeugt. Zu welcher Klasse von Verfahren gehört der hier verwendete Algorithmus? (Siehe Skriptum Kapitel x – bitte Kapitel suchen und Nummer „x“ angeben).}

\end{document}
\grid
